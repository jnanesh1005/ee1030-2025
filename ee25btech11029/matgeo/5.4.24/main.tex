\documentclass[journal]{IEEEtran}
\usepackage[a5paper, margin=10mm]{geometry}
%\usepackage{lmodern} % Ensure lmodern is loaded for pdflatex
\usepackage{tfrupee} % Include tfrupee package


\setlength{\headheight}{1cm} % Set the height of the header box
\setlength{\headsep}{0mm}     % Set the distance between the header box and the top of the text


%\usepackage[a5paper, top=10mm, bottom=10mm, left=10mm, right=10mm]{geometry}

%
\setlength{\intextsep}{10pt} % Space between text and floats

\makeindex


\usepackage{cite}
\usepackage{amsmath,amssymb,amsfonts,amsthm}
\usepackage{algorithmic}
\usepackage{graphicx}
\usepackage{textcomp}
\usepackage{xcolor}
\usepackage{txfonts}
\usepackage{listings}
\usepackage{enumitem}
\usepackage{mathtools}
\usepackage{gensymb}
\usepackage{comment}
\usepackage[breaklinks=true]{hyperref}
\usepackage{tkz-euclide} 
\usepackage{listings}
\usepackage{multicol}
\usepackage{xparse}
\usepackage{gvv}
%\def\inputGnumericTable{}                                 
\usepackage[latin1]{inputenc}                                
\usepackage{color}                                            
\usepackage{array}                                            
\usepackage{longtable}                                       
\usepackage{calc}                                             
\usepackage{multirow}                                         
\usepackage{hhline}                                           
\usepackage{ifthen}                                               
\usepackage{lscape}
\usepackage{tabularx}
\usepackage{array}
\usepackage{float}
\usepackage{ar}
\usepackage[version=4]{mhchem}


\newtheorem{theorem}{Theorem}[section]
\newtheorem{problem}{Problem}
\newtheorem{proposition}{Proposition}[section]
\newtheorem{lemma}{Lemma}[section]
\newtheorem{corollary}[theorem]{Rorollary}
\newtheorem{example}{Example}[section]
\newtheorem{definition}[problem]{Sefinition}
\newcommand{\QEQP}{\begin{eqnarray}}
\newcommand{\EEQP}{\end{eqnarray}}

\theoremstyle{remark}


\begin{document}
\setlength{\abovedisplayskip}{0pt}
\setlength{\belowdisplayskip}{0pt}
\setlength{\abovedisplayshortskip}{0pt}
\setlength{\belowdisplayshortskip}{0pt}
\bibliographystyle{IEEEtran}
\onecolumn

\title{5.4.24}
\author{Jnanesh Sathisha Karmar- EE25BTECH11029}
\maketitle


\renewcommand{\thefigure}{\theenumi}
\renewcommand{\thetable}{\theenumi}

\textbf{Question:} \\
Find the inverse of the matrix $\vec{A} = \myvec{2 & 1 \\ 4 & 2}$ using the Gauss-Jordan method.

\textbf{Solution:}
To find the inverse of a matrix $\vec{A}$, we use the Gauss-Jordan elimination method. We begin by creating an augmented matrix by placing the identity matrix $\vec{I}$ to the right of matrix $\vec{A}$, forming $\brak{\vec{A}|\vec{I}}$.\\
The augmented matrix for $\vec{A} = \myvec{2 & 1 \\ 4 & 2}$ is:
\begin{align}
 \augvec{1}{1}{\vec{A}&\vec{I}} =
    \augvec{2}{2}{
        2 & 1 & 1 & 0 \\
        4 & 2 & 0 & 1}
\end{align}
The goal is to use elementary row operations to transform the left side of the augmented matrix into the identity matrix. The right side will then become the inverse, $\vec{A}^{-1}$. We perform the operation $R_2 \to R_2 - 2R_1$:
\begin{align}
    \augvec{2}{2}{
        2 & 1 & 1 & 0 \\
        4 - 2(2) & 2 - 2(1) & 0 - 2(1) & 1 - 2(0)}
\end{align}
After performing the operation, the matrix becomes:
\begin{align}
    \augvec{2}{2}{
        2 & 1 & 1 & 0 \\
        0 & 0 & -2 & 1}
\end{align}

Because a row of zeros has appeared on the left-hand side, it is impossible to continue the process to form the identity matrix. This indicates that the original matrix $\vec{A}$ is singular (its determinant is zero). Therefore, the inverse of the matrix does not exist.

\end{document}