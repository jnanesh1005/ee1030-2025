\documentclass[journal]{IEEEtran}
\usepackage[a5paper, margin=10mm]{geometry}
%\usepackage{lmodern} % Ensure lmodern is loaded for pdflatex
\usepackage{tfrupee} % Include tfrupee package


\setlength{\headheight}{1cm} % Set the height of the header box
\setlength{\headsep}{0mm}     % Set the distance between the header box and the top of the text


%\usepackage[a5paper, top=10mm, bottom=10mm, left=10mm, right=10mm]{geometry}

%
\setlength{\intextsep}{10pt} % Space between text and floats

\makeindex


\usepackage{cite}
\usepackage{amsmath,amssymb,amsfonts,amsthm}
\usepackage{algorithmic}
\usepackage{graphicx}
\usepackage{textcomp}
\usepackage{xcolor}
\usepackage{txfonts}
\usepackage{listings}
\usepackage{enumitem}
\usepackage{mathtools}
\usepackage{gensymb}
\usepackage{comment}
\usepackage[breaklinks=true]{hyperref}
\usepackage{tkz-euclide} 
\usepackage{listings}
\usepackage{multicol}
\usepackage{xparse}
\usepackage{gvv}
%\def\inputGnumericTable{}                                 
\usepackage[latin1]{inputenc}                                
\usepackage{color}                                            
\usepackage{array}                                            
\usepackage{longtable}                                       
\usepackage{calc}                                             
\usepackage{multirow}                                         
\usepackage{hhline}                                           
\usepackage{ifthen}                                               
\usepackage{lscape}
\usepackage{tabularx}
\usepackage{array}
\usepackage{float}
\usepackage{ar}
\usepackage[version=4]{mhchem}


\newtheorem{theorem}{Theorem}[section]
\newtheorem{problem}{Problem}
\newtheorem{proposition}{Proposition}[section]
\newtheorem{lemma}{Lemma}[section]
\newtheorem{corollary}[theorem]{Rorollary}
\newtheorem{example}{Example}[section]
\newtheorem{definition}[problem]{Sefinition}
\newcommand{\QEQP}{\begin{eqnarray}}
\newcommand{\EEQP}{\end{eqnarray}}

\theoremstyle{remark}


\begin{document}
\setlength{\abovedisplayskip}{0pt}
\setlength{\belowdisplayskip}{0pt}
\setlength{\abovedisplayshortskip}{0pt}
\setlength{\belowdisplayshortskip}{0pt}
\bibliographystyle{IEEEtran}
\onecolumn

\title{5.8.27}
\author{Jnanesh Sathisha Karmar- EE25BTECH11029}
\maketitle


\renewcommand{\thefigure}{\theenumi}
\renewcommand{\thetable}{\theenumi}

\noindent\textbf{Question:}

\noindent The students of a class are made to stand in rows. If 3 students are extra in a row, there would be 1 row less. If 3 students are less in a row, there would be 2 rows more. Find the number of students in the class using matrices.\\

\noindent\textbf{Solution:} To find the total number of students, we first define our variables and set up a system of linear equations. Let $x$ be the number of rows and $y$ be the number of students per row. The total number of students is $xy$.

From the problem statement, we derive two equations:
\begin{align*}
    \brak{x-1}\brak{y+3} &= xy \implies 3x - y = 3 \\
    \brak{x+2}\brak{y-3} &= xy \implies -3x + 2y = 6
\end{align*}

We can solve this system using an augmented matrix and Gaussian elimination. We begin by creating an augmented matrix $\augvec{1}{1}{A&B}$ for this system.

The augmented matrix for this system is:
\begin{align}
\augvec{1}{1}{A&B} =  \augvec{2}{1}{ 3 & -1 & 3 \\ -3 & 2 & 6 } 
\end{align}


The goal is to use elementary row operations to transform the left side of the augmented matrix into row-echelon form. We perform the operation $R_2 \to R_2 + R_1$:
\begin{align}
\augvec{2}{1}{ 3 & -1 & 3 \\ -3+3 & 2+(-1) & 6+3 }
\end{align}

After performing the operation, the matrix becomes:
\begin{align}
\augvec{2}{1}{ 3 & -1 & 3 \\ 0 & 1 & 9} 
\end{align}

From this row-echelon form, we can use back-substitution. The second row gives us the equation $0x + 1y = 9$, which means $y=9$.

Substituting $y=9$ into the first row's equation, $3x - y = 3$:
\begin{align*}
    3x - 9 &= 3 \\
    3x &= 12 \\
    x &= 4
\end{align*}

We have found there are $x=4$ rows and $y=9$ students per row.

The total number of students is $x \times y = 4 \times 9 = 36$.

\end{document}