\documentclass{beamer}
\usepackage[utf8]{inputenc}

\usetheme{Madrid}
\usecolortheme{default}
\usepackage{amsmath,amssymb,amsfonts,amsthm}
\usepackage{txfonts}
\usepackage{tkz-euclide}
\usepackage{listings}
\usepackage{adjustbox}
\usepackage{array}
\usepackage{tabularx}
\usepackage{gvv}
\usepackage{lmodern}
\usepackage{circuitikz}
\usepackage{tikz}
\usepackage{graphicx}
\usepackage{multicol}

\setbeamertemplate{page number in head/foot}[totalframenumber]

\usepackage{tcolorbox}
\tcbuselibrary{minted,breakable,xparse,skins}



\definecolor{bg}{gray}{0.95}
\DeclareTCBListing{mintedbox}{O{}m!O{}}{%
  breakable=true,
  listing engine=minted,
  listing only,
  minted language=#2,
  minted style=default,
  minted options={%
    linenos,
    gobble=0,
    breaklines=true,
    breakafter=,,
    fontsize=\small,
    numbersep=8pt,
    #1},
  boxsep=0pt,
  left skip=0pt,
  right skip=0pt,
  left=25pt,
  right=0pt,
  top=3pt,
  bottom=3pt,
  arc=5pt,
  leftrule=0pt,
  rightrule=0pt,
  bottomrule=2pt,
  toprule=2pt,
  colback=bg,
  colframe=orange!70,
  enhanced,
  overlay={%
    \begin{tcbclipinterior}
    \fill[orange!20!white] (frame.south west) rectangle ([xshift=20pt]frame.north west);
    \end{tcbclipinterior}},
  #3,
}
\lstset{
    language=C,
    basicstyle=\ttfamily\small,
    keywordstyle=\color{blue},
    stringstyle=\color{orange},
    commentstyle=\color{green!60!black},
    numbers=left,
    numberstyle=\tiny\color{gray},
    breaklines=true,
    showstringspaces=false,
}
%------------------------------------------------------------
%This block of code defines the information to appear in the
%Title page
\title %optional
{5.8.27}
\date{October 11,2025}


\author 
{Jnanesh Sathisha karmar - EE25BTECH11029}



\begin{document}



\frame{\titlepage}
\begin{frame}{Question:}


\noindent The students of a class are made to stand in rows. If 3 students are extra in a row, there would be 1 row less. If 3 students are less in a row, there would be 2 rows more. Find the number of students in the class using matrices.\\
\end{frame}

\begin{frame}{Theoretical Solution}
To find the total number of students, we first define our variables and set up a system of linear equations. Let $x$ be the number of rows and $y$ be the number of students per row. The total number of students is $xy$.

From the problem statement, we derive two equations:
\begin{align*}
    \brak{x-1}\brak{y+3} &= xy \implies 3x - y = 3 \\
    \brak{x+2}\brak{y-3} &= xy \implies -3x + 2y = 6
\end{align*}

We can solve this system using an augmented matrix and Gaussian elimination. We begin by creating an augmented matrix $\augvec{1}{1}{A&B}$ for this system.

The augmented matrix for this system is:
\begin{align}
\augvec{1}{1}{A&B} =  \augvec{2}{1}{ 3 & -1 & 3 \\ -3 & 2 & 6 } 
\end{align}
\end{frame}
\begin{frame}{Theoretical Solution}
The goal is to use elementary row operations to transform the left side of the augmented matrix into row-echelon form. We perform the operation $R_2 \to R_2 + R_1$:
\begin{align}
\augvec{2}{1}{ 3 & -1 & 3 \\ -3+3 & 2+(-1) & 6+3 }
\end{align}

After performing the operation, the matrix becomes:
\begin{align}
\augvec{2}{1}{ 3 & -1 & 3 \\ 0 & 1 & 9} 
\end{align}

From this row-echelon form, we can use back-substitution. The second row gives us the equation $0x + 1y = 9$, which means $y=9$.
Substituting $y=9$ into the first row's equation, $3x - y = 3$:
\begin{align*}
    3x - 9 &= 3 \\
    3x &= 12 \\
    x &= 4
\end{align*}


\end{frame}
\begin{frame}{Theoretical Solution}
We have found there are $x=4$ rows and $y=9$ students per row.

The total number of students is $x \times y = 4 \times 9 = 36$.

    
\end{frame}

\begin{frame}[fragile]
    \frametitle{C Code }

    \begin{lstlisting}
#include <stdio.h>

void print_matrix(float matrix[2][3]) {
    for (int i = 0; i < 2; i++) {
        for (int j = 0; j < 3; j++) {
            printf("%8.2f ", matrix[i][j]);
        }
        printf("\n");
    }
}

int main() {
    float matrix[2][3] = {
        {3.0, -1.0, 3.0},
        {-3.0, 2.0, 6.0}
    };

    printf("Initial Augmented Matrix:\n");
    print_matrix(matrix);


    \end{lstlisting}
\end{frame}

\begin{frame}[fragile]
    \frametitle{C Code }
    \begin{lstlisting}
    
    for (int j = 0; j < 3; j++) {
        matrix[1][j] = matrix[1][j] + matrix[0][j];
    }
    printf("\nMatrix after Row Operation (R2 -> R2 + R1):\n");
    print_matrix(matrix);
    float y = matrix[1][2] / matrix[1][1];
    float x = (matrix[0][2] - (matrix[0][1] * y)) / matrix[0][0];
    int total_students = (int)(x * y);

    printf("\n--- Solution ---\n");
    printf("Number of rows (x): %.0f\n", x);
    printf("Number of students per row (y): %.0f\n", y);
    printf("Total number of students in the class: %d\n", total_students);

    return 0;
}
    return 1;
\end{lstlisting}
\end{frame}




\end{document}